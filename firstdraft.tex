% This is the first draft of MMU Faculty of Engineering Cryptography and Security systems assignment
% for 18/19 Trimester 2
% Chia Jason 1161300548
% Hor Sui Lyn 1161300122
%
\documentclass[runningheads]{llncs}
\usepackage{graphicx}
\usepackage[T1]{fontenc}
\usepackage[utf8]{inputenc}
\usepackage[portuguese]{babel}
\usepackage{hyphenat}
\hyphenation{mate-mática recu-perar}
%
\begin{document}
% title
\title{Modelando dados de incidência de câncer de próstata}
%\thanks{Supported by Mult x.} % if we want to thank someone/org.
%
% authors
\author{Jailson Rodrigues}
%
\authorrunning{C. Jason and H. Sui Lyn}
% First names are abbreviated in the running head.
%
\institute{{Multimedia University, Cyberjaya 63100, Malaysia}}
%
\maketitle              % typeset the header of the contribution
%
\begin{abstract}
The attendance system at Multimedia University has evolved from signing on papers to a QR code based attendance system, whereby students were to scan the QR code projected by a lecturer to register their attendance for the class. This paper examines the approach of a QR code based attendance system and proposes a method to exploit the oversight of it's implementation. 
%
\keywords{Attendance System  \and Parameter tampering \and Exploit discovery.}
\end{abstract}
%
%
%
\section{Introdução}
\subsection{Quick Response (QR) Code based attendance system}
Um grupo de pesquisadores de um determinado centro médico universitário está interessado em estudar a associação entre antígeno específico da próstata (PSA) e algumas medidas clínicas prognósticas em homens com câncer de próstata em estado avançado. Os dados foram coletados de 97 homens que estavam prestes a sofrer prostatectomias radicais. O conjunto de dados possui um número identificando o paciente e informações a respeito de 8 medidas clínicas.


%
%
\section{Descrição dos dados}
\subsection{Cross-Site Request Forgery} 
\begin{table}[!htp]
\begin{tabular}{c|ll}
\hline
Número da variável & Nome da variável               & Descrição                                                                                                                     \\ \hline
1                  & Número de identificação        & 1-97                                                                                                                          \\
2                  & Nível PSA                      & \begin{tabular}[c]{@{}l@{}}Nível sérico de antígeno prostático específico\\   (mg/ml)\end{tabular}                            \\
3                  & Volume câncer                  & Estimativa do volume do câncer (cc)                                                                                           \\
4                  & Peso                           & Peso da próstata (gm)                                                                                                         \\
5                  & Idade                          & Idade do paciente (anos)                                                                                                      \\
6                  & Hiperplasia prostática benigna & Quantidade de hiperplasia prostática benigna (cm²)                                                                            \\
7                  & Invasão da vesícula seminal    & Presença ou ausência (1 se sim; 0 se não)                                                                                     \\
8                  & Penetração capsular            & Grau de penetração capsular (cm)                                                                                              \\
9                  & Escore Gleason                 & \begin{tabular}[c]{@{}l@{}}Grau patologicamente\\ determinado da doença \\(escores altos indicam pior prognóstico)
\end{tabular}
\end{tabular}
\end{table}
%
%

\end{document}